\documentclass[lettersize,journal]{IEEEtran}
\usepackage{amsmath,amsfonts}
\usepackage{algorithmic}
\usepackage{algorithm}
\usepackage{array}
\usepackage[caption=false,font=normalsize,labelfont=sf,textfont=sf]{subfig}
\usepackage{textcomp}
\usepackage{stfloats}
\usepackage{url}
\usepackage{verbatim}
\usepackage{graphicx}
\usepackage{cite}
\usepackage{hyperref}

\hyphenation{op-tical net-works semi-conduc-tor IEEE-Xplore}
% updated with editorial comments 8/9/2021

\begin{document}

\title{A Sample Article Using IEEEtran.cls\\ for IEEE Journals and Transactions}

\author{IEEE Publication Technology,~\IEEEmembership{Staff,~IEEE,}
        % <-this % stops a space
\thanks{This paper was produced by the IEEE Publication Technology Group. They are in Piscataway, NJ.}% <-this % stops a space
\thanks{Manuscript received April 19, 2021; revised August 16, 2021.}}

% The paper headers
\markboth{Journal of \LaTeX\ Class Files,~Vol.~14, No.~8, August~2021}%
{Shell \MakeLowercase{\textit{et al.}}: A Sample Article Using IEEEtran.cls for IEEE Journals}

\IEEEpubid{0000--0000/00\$00.00~\copyright~2021 IEEE}
% Remember, if you use this you must call \IEEEpubidadjcol in the second
% column for its text to clear the IEEEpubid mark.

\maketitle

\begin{abstract}
This document describes the most common article elements and how to use the IEEEtran class with \LaTeX \ to produce files that are suitable for submission to the IEEE.  IEEEtran can produce conference, journal, and technical note (correspondence) papers with a suitable choice of class options. 
\end{abstract}

\begin{IEEEkeywords}
Article submission, IEEE, IEEEtran, journal, \LaTeX, paper, template, typesetting.
\end{IEEEkeywords}

\section{Introduction}
\IEEEPARstart{A}{s} per studies, one-third of music streaming on smartphones is intended to influence affective states \cite{wadley2019use}. According to a recent survey, 73.8\% of respondents reported using online music as a means of managing their emotions \cite{martin2021music}. Using social media applications was the most effective technique to self-regulate mood during the COVID-19 pandemic. Pre-pandemic, a different survey found that 87\% of teens and adults would go online for ways to deal with stress, anxiety, or depression, but just 20\% of them would consider seeking professional help \cite{rideout2018digital}. Some instances of emotions driving smartphone usage in daily life include listening to music while exercising, watching online videos during daily commutes, scrolling through social media while waiting in a queue, and browsing shopping websites at the workplace.


The practice of consciously modifying one's affective state is called emotion regulation (ER). The ability to successfully perform emotion regulation is essential to function effectively in everyday life, to act appropriately in everyday interactions, or merely for hedonic purposes. The ubiquity of digital technology has produced a plethora of prospects for understanding, managing and influencing the world we live in. Today, digital technology is being used as a tool to strategically influence our affective states (such as emotions, mood and stress levels), a process known as digital emotion regulation (DER). In addition to its novelty and scope, understanding digital emotion regulation can facilitate the advancement of the ethical design, development, and application of technology.


The rise in popularity of social media applications has led to significant virtualisation of engagement activities in our lives. Since these applications provide a variety of dimensions for expression and consumption, traditional forms of leisure and amusement have also been revolutionised by this digitisation. The COVID-19 pandemic made it difficult to socialise offline, thus more people than ever are choosing to spend their social lives online. Emotions are intertwined in our daily affairs and shape our interactions because they are an integral part of human behaviour and influence how we think and act. Our experiences elicit emotional responses, and these emotions frame our actions. The same is also true for digital technology and social media. It has been observed that emotional states impact interactions with technology, such as joystick movements, typing errors, as well as typing speed. Recent studies have established a correlation between emotional states and the number of application launches, duration of application use, and type of application used, demonstrating that the relationship between emotions and social media use is bidirectional \cite{sarsenbayeva2020does}. People have created a toolkit that allows them to actively appropriate and integrate a variety of applications to manage emotions in everyday life \cite{smith2022digital}. Scrolling through social media feeds for distraction, watching cat videos for stress relief, texting a friend to seek support, following an inspirational person to focus on success, and sharing emotions through status updates are a few examples of emotion regulation via social media platforms. The omnipresence of social media applications has optimised our ability to regulate emotions at any time or place by allowing us to fine-tune our strategies for specific situations and engage in emotion regulation more frequently.

Understanding emotion regulation necessitates learning about the process of emotion regulation as well as the motivation for doing so. Gross's process model describes how people regulate their emotions by classifying the strategies for emotion regulation into five categories based on when they are encountered in the sequence of emotion generation \cite{wadley2020digital}. Situation selection is the first intervention, which entails going into situations that might evoke desired emotions or staying away from scenarios that might induce undesirable emotions. Situation modification makes it possible to control emotions by directly altering specific aspects of a situation after it has been encountered, however before an emotional reaction has fully formed. This is accomplished by using attentional deployment to shift one's focus from or toward emotion-relevant features, or by reevaluating a situation to change the way its emotional impact through cognitive change. Lastly, response modulation can be used to change the physiological, behavioural, or experiential components of an emotional response even after it has already occurred. For instance, by exhibiting intensified facial expressions. Tamir's motive taxonomy, which divides the goal of emotion regulation into two categories, hedonic and instrumental, with additional classifications, may be used to explain emotion regulation motivations. Hedonic motives (Prohedonic and Contrahedonic) entail engaging in emotion regulation in order to experience or avoid certain emotions; typically, pleasant emotions are enhanced while painful emotions are diminished. Instrumental motives (Performance, Epistemic, and Social) are motivations that people have when they believe that certain affective states will help them achieve a goal, meet performance objectives, or demonstrate socially acceptable expressions and behaviours \cite{wadley2020digital}.

The examples discussed thus far in this article are cases of intrinsic ER, in which people intend to alter their own affective states; however, in digital technologies, extrinsic ER is an essential component, where people intentionally try to influence the emotional state of others. Social media platforms allow users to create identifiable profiles, represent public connections and consume, produce, and/or interact with a wide range of user-generated content created by their connections on the site. These applications consistently harbour communication among millions of people from all over the world and hence contain traces of emotional expression in abundance. Since the virtual world offers an opportunity to network with people beyond geographical boundaries, exposure to the expressions of others amplifies, which has the potential to influence the emotions, actions and online behaviours of the users. Because user engagement is a key factor in the success of social media platforms, and exposure to emotion increases it, social media companies have strategically used the emotional effects of activities on social media platforms to improve user experience and fuel user participation. This intervention by digital media companies to boost the intensity and frequency of expression has resulted in a rise in online emotion contagion, which is the spontaneous spread and convergence of emotion based on exposure \cite{goldenberg2020digital}. According to ER psychological research, emotion contagion can be caused by extrinsic emotion regulation \cite{elfenbein2014many}. Online movements such as \#MeToo and toilet paper hoarding during COVID-19, which rely on users' connected behaviours and have an impact on users' offline lives as well, emerged as a result of emotional contagion. Implicit emotion regulation, which does not involve a conscious desire to alter emotional responses, may also play a role in the development of connective action. Nonetheless, through the disproportionate induction and convergence of emotion, it generates social dysfunction and online toxicity.


There is still a substantial difference in social behaviour between online and offline, even though as technology becomes a more accepted part of our life, the boundary between the real and the virtual is dissolving. Although online toxicity has been well understood, recognised, and intensively studied for a few years now, combatting measures are challenging to implement due to the nature of online media environments, where the context of user behaviours cannot always be extracted. Anonymity in social media platforms keeps the users behind the safety of a keyboard. This lack of accountability in social media platforms has resulted in poverty in online well-being. What is acceptable online (due to lack of mediation, moderation, or other restrictions), may not be accepted offline. This emphasises the importance of researching how emotion regulation could be utilized to ensure consistent well-being online by enabling ethical design and development in social media applications. As a result, this work sheds light on recent advances in DER for social media applications. It focuses on the applications of identifying and understanding emotion regulation on these platforms by providing a comparative analysis of emotion regulation strategies used by users on various social media applications. Therefore, the following are the main contributions of this work:
\begin{itemize}
    \item We provide an overview of digital emotion regulation in social media applications, as well as a synthesis of recent research on emotion regulation interventions for social media.
    \item We share our findings from analysing a variety of literature articles on how different social media applications are used at various stages of the emotion regulation process.
    \item We offer a theoretical analysis of the extrinsic emotion regulation strategies prevalent on social media platforms which may help improve online well-being.
\end{itemize}

The rest of this article is structured as follows. We begin by discussing the evolution of digital emotion regulation in social media platforms, followed by a description of the popular approaches used to analyse emotions in social media platforms (the 'how' and 'what'). Then we present some applications of understanding emotional regulation in social media applications, as well as methods for implementing them. Subsequently, we present a comparative analysis of emotion regulation strategies employed by users on various social media applications. Finally, we identify gaps in the literature and conclude the article with some future directions.


\section{Emotion Regulation in Social Media}
HCI research has seen an increase in interest in the design and development of technological interventions that can help and support emotion regulation. This renewed interest originates from the realisation and recognition that effective emotion regulation is a vital aspect of mental health. Over the last decade, there has been a lot of research into technology-enabled mental health support and the development of a diverse array of tools designed to assist with emotion regulation. The availability and popularity of low-cost wearable sensors and extensively configurable bio-feedback devices fueled this. Another factor that has contributed to this advancement is psychological research in the field of emotion regulation, which now recognises effective ER as a key factor in personal well-being and is using it as a trans-diagnostic intervention for mental health disorders. This section summarises the recent elements of research and development that have emerged in the field of digital emotion regulation. 

\subsection{Observational Studies}
Recent studies have examined how people combine a variety of social media applications, such as video streaming platforms, discussion forums, online games, and music, to regulate their emotions, thoughts, and behaviours. A common technique known as "mental reset" has been observed, in which individuals employ social media apps to distract themselves from overwhelming emotions. Several diary studies have been conducted to better understand various aspects of everyday emotion regulation in isolation, such as the use of social media to avert homesickness and university students' use of music streaming platforms. Individuals' multitasking and passive scrolling habits on social media apps have also been explored in studies revealing how people voluntarily take breaks from social media to mitigate negativity or uphold a sense of equilibrium, as well as highlighting the practice of interpersonal emotion regulation in discussion forums \cite{smith2022digital}. It was discovered that active social media usage had a higher rate of emotion regulation than passive usage, which provided superior support for rest and recovery \cite{hossain2022motivational}. The effect of the pandemic on young people's digital emotion regulation habits has also been studied, and it indicates that while the lockdown made people's emotion regulation practises more uniform, it also increased their reliance on technology and increased their proclivity to use emotion suppression strategies \cite{tag2022impact}. Computing research in digital emotion regulation draws on psychological research literature by employing Gross's process model for classifying ER strategies and practices, as well as Tamir's motives for identifying the goals of performing ER. Altogether, these studies show the variety of digital technologies people use to regulate their emotions in their daily lives. They emphasise the importance of the technologies packed into these devices and the need to boost well-being online, in addition to the growing significance of digital technologies in promoting mental well-being on a global scale.
\subsection{Design and Evaluation of Novel Tools}
This body of research involves the development of interventions that seek to support, develop, or guide emotion regulation skills or help individuals use such skills in challenging situations. The process of emotion regulation occurs in four stages: identifying the need to regulate emotions, selecting an appropriate strategy, implementing it, and then monitoring the regulated state to determine whether additional regulation is required. Technology-enabled interventions have aimed at either supporting a specific ER strategy or boosting emotional awareness during the identification or monitoring stages \cite{slovak2022designing}. They include experience-based design components which primarily focus on bio-feedback or implicit target responses to nudge users subconsciously toward specific physiological states via haptic interaction, such as such as imitating heart rate to boost performance by decreasing anxiety \cite{smith2022digital}. These involve features for both on-the-spot and offline support. On-the-spot support systems use triggers to guide or support users' during the emotion regulation processes, and they can provide either one-time reminders or ongoing haptic support. Offline support mechanisms use interactivity to generate a personalised emotion regulation cycle, such as visualising a timeline of emotional patterns to structure users' reflections. Recent advancements have also seen didactic intervention components that rely on reminder-based recommender systems, such as suggesting specific ER strategies to users and facilitating emotion awareness by prompting users to consider how they feel or felt. These works investigate new design possibilities for DER, and by examining how these designs affect users, they open up new avenues for research in this field.

\subsection{Recognising Emotion Regulation}
Emotion Regulation entails a person, their environment, a situation that generates emotion, and the person's efforts to regulate that emotion. Measuring these variables in a lab is difficult, so studies have begun investigating ways to recognise them in the wild. To observe the change of emotions using facial expressions, the front camera of smartphones, touch sensors, eye trackers, and motion sensors have been used, both independently and in combination. A recent study used the device's front camera to measure the degree of joy throughout each phone session and divided people into three groups based on how likely they were to feel joy at the start of a session \cite{tag2022emotion}. Go-getters were users who desired a quick experience of joy before unlocking their phone, targeters were users who desired a uniform and gradually increasing joy as the session progressed, and explorers were users who experienced the polymodal joy that sees a gradual decline before locking the phone. Another study applied image manipulation to simulate the realistic image distortion that occurs when capturing a person's expressions for facial expression-based detection of digital emotion regulation \cite{yang2021benchmarking}. According to research, combining modalities improves the accuracy of affect detection in social media tasks.


\section{Approaches to Study Digital Emotion Regulation in Social Media}
Research in digital emotion regulation requires observing the entire process of emotion regulation, beginning with recognising the need to regulate, evaluating the context, and implementing, monitoring, and modifying strategies. The majority of psychological research in ER is based on static self-reported assessments, experience sampling, surveys, or lab-based tasks collected via questionnaires, interviews, and group discussions, making the state-sensitive task of emotion regulation difficult to capture. Scholars have been motivated to investigate how people use digital media for this process since the recent realisation of the emergence of DER. Emotion sensing has gained popularity in HCI research, implying that everyday ubiquitous technology can be effectively utilised for emotion detection. The various sensors (such as accelerometer, gyroscope, proximity sensor, and microphone) built into devices can be used to detect users' internal and external contexts, including emotions. The above tools allow us to gain a better understanding of the state-sensitive processes by which people utilise appropriate emotion-regulation strategies in their everyday life. A growing number of technology devices capable of both passive and active analysis, such as these, are being used to measure specific emotion-regulation processes. Digital technologies, in particular, allow for the comprehensive analysis of instances where individuals engage in regulation strategies, the ways in which they apply or adjust these techniques, as well as how well they work in various circumstances and over time.


Most of the studies exploring the use of social media for emotion regulation have used self-reported questionnaires or a diary-keeping technique. In this type of data collection, participants are expected to record their interactions with emotion regulation and technology use over a set period of time, followed by an interview about their technology use and associated affective responses. This method limits the richness of what can be captured to a minimum, allowing participants to trace and reflect on meaningful insights in both work and social life. Users' well-being on social media was investigated by analysing the expressible, shareable, consumable, and evaluable emotional affordances of social media technologies, which revealed that all of these emotional affordances yield emotions and responses associated with emotions that have a substantial influence on digital well-being but do not play a role in eudaimonic well-being. 


Methods of exploring DER in social media by combining a smartphone's front camera and touch sensor have also become popular. This data has been used to predict the participants' binary emotional states and can also detect the emotional state while passively using social media. The front camera has been used to measure the degree of joy for a phone session to discover unimodal and polymodal patterns of joy. Researchers contend that by using manipulated images that simulate the realistic image distortion that occurs when a person's expressions are captured while using their phone, recognition systems rarely confused expressions of surprise, anger, happiness, and neutrality for other emotions. Attention-based LSTM has also been proposed for a user-independent mobile emotion recognition system that uses smartphone-only or wristband+smartphone data.

Emotions are inextricably linked to user well-being, according to research, and if these affordances are exploited, our digital well-being is particularly vulnerable. It is alleged that, while social anxiety does not cause problematic social media and smartphone usage patterns on its own, individuals with high levels of social anxiety may prefer online media over face-to-face interactions, which could be the cause of the indirect pathway linking affection-related distorted cognitions and social anxiety.  When online content creators were interviewed to learn about their coping strategies and personal experiences with attacks, it was discovered that content moderation was the most effective approach of all. According to 65\% of creators, platform regulations and community rules were at least somewhat effective in preventing harassment and hate crimes. They advocate for the informed use of smartphones for effective emotion regulation through the development of methods that improve emotional and social competence. They argue that a better understanding of these attacks can help prevent them from becoming serious sources of harm, such as stalking or violent threats.





\section{Applications of Studying Digital Emotion Regulation for Social Media}
Because social media forms such a massive part of digital technology, it has inevitably become a part of digital media regulation due to the diverse range of expression opportunities it provides. However, the ease with which emotions can be regulated through digital media does not guarantee their effectiveness, as we see online rumination, trolling, and emotional effusion, which may be caused by emotion dysregulation. Understanding how people use social media platforms to regulate their emotions will help to shape the design and development of strategies to promote effective emotion regulation and improve online well-being. Because of the popularity and widespread use of social media applications,  \href{https://www.genroe.com/blog/social-media-statistics-australia/13492
}{(82.4\% of Australians use social media and spend an average of about 2 hours per day online)}, they can be used to deliver ER-based learning as a takeaway. Today's content moderation is based on identifying specific words or phrases in a post or comment. Understanding how to recognise emotion regulation in online activities will also provide a factual foundation for content moderation.
\subsection{Enhance Online Well-Being}
For the past few years, there has been debate about online toxicity and the negative effects of digital technologies. The idea that emotion can be influenced by technology, that it plays a significant role in the user experience, and that it can be catered to through design is a fundamental component of contemporary HCI. Research on user interface design aims to support favourable emotional interactions with digital tools and services, whereas affective computing research aims to detect and adapt to user emotions. Psychology lessons have recently been applied to HCI to develop applications that support and enhance emotional well-being. Digital emotion regulation will not only help understand the relationship between emotions and technology use but will also inform the conceptualization of emotion online, which will eventually enable the creation of services that enrich online well-being.
\subsection{Inform Content Moderation}
The vast majority of DER research has been conducted in controlled lab settings, with observation data derived from diary studies, interviews, experience sampling, and remote sensing. These techniques provide insights into the timeline of changes in user experience while also capturing the rich context-dependency and complexity of emotions in their natural state. The sensitive nature of this data, however, makes it impractical for sharing and ineffective for the research community, preventing further development in the area. It has been identified that people use social media as an ER tool. Social media contains a wealth of data containing the emotional trajectories of users worldwide. As a result, exploring ER for social media platforms will help in determining emotion regulation in online activities, which will eventually inform strategies for moderation and control of emotion dysregulation online. Because social media data is easily accessible, it will stimulate contrast and comparison.

\subsection{Emotion Regulation Transfer Support}
The transfer of learning is a crucial part of ER interventions. The long-term impact and sustainability of efficient systems depend on people being able to manage daily situations, for instance, without ongoing support. The ER technologies currently include applications that serve as reminders and didactic tools (module-based). Few experiential applications that provide bio-feedback focus on one or more ER strategies (usually suppression, which is not considered ideal for long-term use), which makes the support for ER strategies inconsistent. Understanding and recognising ER in social media activities will make it easier to determine the emotion regulation intervention goals that systems intended to support ER may aim to impact, as well as how/when the interventions can play a role in the ER process. The 'how' here refers to the mode of delivery, which could be didactic or experiential, whereas the 'when' refers to whether the delivery is done offline or on the spot. Examining how people regulate their emotions on social media will empower researchers in developing novel ways to reinforce theory-driven ER targets using novel intervention delivery methods.


\begin{figure}[!t]
\centering
\includegraphics[width=2.5in]{fig1}
\caption{Simulation results for the network.}
\label{fig_1}
\end{figure}


\begin{figure*}[!t]
\centering
\subfloat[]{\includegraphics[width=2.5in]{fig1}%
\label{fig_first_case}}
\hfil
\subfloat[]{\includegraphics[width=2.5in]{fig1}%
\label{fig_second_case}}
\caption{Dae. Ad quatur autat ut porepel itemoles dolor autem fuga. Bus quia con nessunti as remo di quatus non perum que nimus. (a) Case I. (b) Case II.}
\label{fig_sim}
\end{figure*}



 




\section{Conclusion and Future Work}
Since emotion regulation is such an essential component of overall health, it is necessary to develop interventions that can help people shift their dominant (or modified) coping mechanisms to digital forms in ways that are broadly applicable and manualized. The desire to regulate one's emotions has been linked to the problematic use of social media and smartphones. Users spend hours scrolling through social media sites to relieve stress (engaging in unhealthy strategies for emotion regulation like rumination, overthinking or self-blame), but they end up feeling distressed because of the content they consume and the amount of time they waste procrastinating. A lack of emotion regulation skills may be the root cause of problematic internet use. As a result, we believe the following potential future research directions in digital emotion regulation will help bridge the gap between emotions and technology usage.
\subsection{Developing Digital Solutions} Current emotion regulation tools include mood-based recommendation systems and reminders, which can only provide temporary assistance and are difficult to integrate into daily life. Furthermore, studies show that a combination of online media platforms and applications are used to perform digital emotion regulation; however, these applications or platforms have not been curated for the purpose. Innovative digital solutions that support the transfer and development of emotion regulation practices will not only have a long-term impact but will also bolster social media usage. Sophisticated tools that allow people to consider, evaluate, and self-diagnose the positive and negative outcomes of their digital activities are needed. There is a lack of techniques that use existing or new mechanisms for interaction design that help users navigate emotional experience trajectories as opposed to providing recommendations based on their emotional state.

\subsection{Synthesising Data and Prototypes for Contrast and Comparison} Since the majority of recent research in the field of digital emotion regulation is based on field studies, ecological momentary assessments (EMA), or physiological sensors combined with facial data, which are not feasible to share due to their sensitive nature, there is a lack of data that can be used to compare and contrast the results produced, as well as a lack of a common prototype for synthesising emotion regulation, as discussed earlier in the article.
\subsection{Understanding as to how digital emotion regulation can be recognised} Although the relevance of situation and context in the process of regulating digital emotions is widely acknowledged, little research in this area has been conducted through field studies and interviews. Understanding a user's online presence and connecting it to their approach to emotion regulation would provide new insights into how technology can help break down and personalise existing cognitive modules into manageable pieces. Empirical studies of how people deploy and use digital technologies for emotional outcomes are required to learn how users can be supported by digital media when undergoing emotion regulation. Because of the complexity of emotional expressions, a large portion of DER research presents qualitative results that are difficult to extend. This opens up opportunities for the use of machine learning and deep learning-based algorithms, which can be used to understand the habits, frequency, and intensity of DER practises online initially.

Researchers have addressed concerns about the negative effects of excessive technology use on mental health. Although knowledge of digital ER does not eliminate such concerns, it does suggest that some digital usage may be part of more constructive psychological techniques. Learning more about digital emotion regulation can add nuance to discussions about the overuse of digital tools by highlighting the possibility that people's motivations for using technology for seemingly counterproductive activities, such as distraction, may not always be hedonistic but instead serve significant instrumental purposes, such as improving work performance or fostering social harmony.


\section*{Acknowledgments}
This should be a simple paragraph before the References to thank those individuals and institutions who have supported your work on this article.



{\appendix[Proof of the Zonklar Equations]
Use $\backslash${\tt{appendix}} if you have a single appendix:
Do not use $\backslash${\tt{section}} anymore after $\backslash${\tt{appendix}}, only $\backslash${\tt{section*}}.
If you have multiple appendixes use $\backslash${\tt{appendices}} then use $\backslash${\tt{section}} to start each appendix.
You must declare a $\backslash${\tt{section}} before using any $\backslash${\tt{subsection}} or using $\backslash${\tt{label}} ($\backslash${\tt{appendices}} by itself
 starts a section numbered zero.)}



%{\appendices
%\section*{Proof of the First Zonklar Equation}
%Appendix one text goes here.
% You can choose not to have a title for an appendix if you want by leaving the argument blank
%\section*{Proof of the Second Zonklar Equation}
%Appendix two text goes here.}



\section{References Section}
You can use a bibliography generated by BibTeX as a .bbl file.
 BibTeX documentation can be easily obtained at:
 http://mirror.ctan.org/biblio/bibtex/contrib/doc/
 The IEEEtran BibTeX style support page is:
 http://www.michaelshell.org/tex/ieeetran/bibtex/
 
 % argument is your BibTeX string definitions and bibliography database(s)
%\bibliography{IEEEabrv,../bib/paper}
%
\section{Simple References}
You can manually copy in the resultant .bbl file and set second argument of $\backslash${\tt{begin}} to the number of references
 (used to reserve space for the reference number labels box).


\bibliographystyle{IEEEtran}
\bibliography{refs.bib}



\newpage

% \section{Biography Section}
% \vspace{11pt}

% \bf{If you include a photo:}\vspace{-33pt}
% \begin{IEEEbiography}[{\includegraphics[width=1in,height=1.25in,clip,keepaspectratio]{fig1}}]{Michael Shell}
% Use $\backslash${\tt{begin\{IEEEbiography\}}} and then for the 1st argument use $\backslash${\tt{includegraphics}} to declare and link the author photo.
% Use the author name as the 3rd argument followed by the biography text.
% \end{IEEEbiography}

% \vspace{11pt}

% \bf{If you will not include a photo:}\vspace{-33pt}
% \begin{IEEEbiographynophoto}{John Doe}
% Use $\backslash${\tt{begin\{IEEEbiographynophoto\}}} and the author name as the argument followed by the biography text.
% \end{IEEEbiographynophoto}




\vfill

\end{document}


